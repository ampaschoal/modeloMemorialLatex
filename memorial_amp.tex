%% Garcia's latex template for Activities Drescriptive Memorial Report
%% Version 0.2
%% (c) 2015 Vinicius Cardoso Garcia (vcg@cin.ufpe.br)
%%
%% This document is based on latex template from Prof. Daniel Cunha.
%%
%% Reference commands. Use the following commands to make references in your
%% text:
%%          \figref  -- for Figure reference
%%          \tabref  -- for Table reference
%%          \eqnref  -- for equation reference
%%          \chapref -- for chapter reference
%%          \secref  -- for section reference
%%          \appref  -- for appendix reference
%%          \axiref  -- for axiom reference
%%          \conjref -- for conjecture reference
%%          \defref  -- for definition reference
%%          \lemref  -- for lemma reference
%%          \theoref -- for theorem reference
%%          \corref  -- for corollary reference
%%          \propref -- for proprosition reference
%%          \pgref   -- for page reference
%%
%%          Example: See \chapref{chap:introduction}
%%%%%%%%%%%%%%%%%%%%%%%%%%%%%%%%%%%%%%%%%%%%%%%%%%%%%%%%%%%%%%%%%%%%%%%%%%%%%%%

\documentclass[a4paper,oneside,10pt]{article}

\usepackage{graphicx}
\usepackage{amsmath,amstext,amssymb,amsfonts}
\usepackage[none]{hyphenat}
\usepackage{fancyhdr}
\usepackage{cite}
\usepackage{indentfirst}
%\usepackage{path}
\usepackage[usenames, dvipsnames]{color}

\usepackage{pdfsync}
\usepackage{url}
\usepackage{setspace}
\usepackage[latin1,utf8]{inputenc}
\usepackage[T1]{fontenc}
\usepackage{colortbl}
\newcommand{\SetRowColor}[1]{\noalign{\gdef\RowColorName{#1}}\rowcolor{\RowColorName}}

\definecolor{MyRed}{rgb}{1,0.2,0.1}
\definecolor{light-gray}{gray}{0.95}
\definecolor{gray}{gray}{0.6}

\usepackage{booktabs}
\usepackage{ctable}
\usepackage{setspace}
\usepackage[multidot]{grffile}
\usepackage[final]{pdfpages}

\usepackage{titlesec}

\usepackage[toc,page]{appendix}
\newcommand{\appref}[1]{\@appendixname~\ref{#1}\xspace}
%\usepackage{xifthen}

\usepackage[bookmarks,colorlinks,pdfpagelabels,
pdftitle={Memorial Descritivo de Atividades}, pdfauthor={André Monteiro Paschoal},
pdfsubject={Solicita\c{c}\~{a}o de progresss\~{a}o funcional docente de Adjunto N\'{\i}vel 3 para Adjunto N\'{\i}vel 4 apresentada \`{a} Comiss\~{a}o de Avalia\c{c}\~{a}o de Progress\~{a}o Horizontal do Centro de Inform\'{a}tica da Universidade Federal de Pernambuco.},
pdfcreator={André Monteiro Paschoal}, pdfkeywords={Professor, Doutor, IFGW, UNICAMP, Docente}]{hyperref}

\newcommand{\otoprule}{\midrule[\heavyrulewidth]}

% Defini\c{c}\~{a}o de margens
\setlength{\textwidth}{16cm} %
\setlength{\textheight}{23cm} %
\setlength{\oddsidemargin}{0cm} %
\setlength{\evensidemargin}{0cm} %
\setlength{\topmargin}{0cm}

\renewcommand{\abstractname}{Resumo}
\renewcommand{\contentsname}{\'{I}ndice Anal\'{\i}tico}
\renewcommand{\refname}{Refer\^{e}ncias}
\renewcommand{\appendixname}{Ap\^{e}ndice}
\renewcommand{\tablename}{Tabela}

%% Formata\c{c}\~{a}o do cabe\c{c}alho e rodap\'{e}
\lhead{\footnotesize Memorial Descritivo de Atividades} %
\rhead{\footnotesize \emph{André Monteiro Paschoal}} %
\chead{} %
\cfoot{} %
\lfoot{\footnotesize\nouppercase\leftmark} %
\rfoot{\bfseries\thepage}
\renewcommand{\footrulewidth}{0.1pt}

% Comando para inserir n\'{u}mero de documento
\newcounter{document}%[section]
\setcounter{document}{0}
\renewcommand\theenumi{\arabic{section}.\arabic{enumi}}
\newcommand\Doc{{\addtocounter{document}{1}\mbox{\sffamily\bfseries [Doc. \arabic{document}]}}}

% Comando para repetir um n\'{u}mero de documento j\'{a} citado
% \mbox{\sffamily{\bfseries{[Doc. XX]}}}

%% Alternativa na edi\c{c}\~{a}o dos comandos.
%% Comando para inserir n\'{u}mero de documento
% \newcommand\thedocument{%
%    \ifthenelse{\arabic{subsection}=0}
%      {\thesection.\arabic{document}}
%      {\thesubsection.\arabic{document}}}
% \newcounter{document}[section]
% \setcounter{document}{0}
% \renewcommand\theenumi{\arabic{section}.\arabic{enumi}}
% \ifthenelse{\arabic{subsection} = 0}{\newcommand\Doc{{\stepcounter{document}\bfseries [Doc. \arabic{section}.\arabic{document}]}}}{\newcommand\Doc{{\stepcounter{document}\bfseries [Doc. \arabic{section}.\arabic{subsection}.\arabic{document}]}}}
% %\newcommand\Doc{{\addtocounter{document}{1}\mbox{\bfseries [Doc. \arabic{document}]}}}


% Ambiente para centralizar vertical
\newenvironment{vcenterpage}
     {\newpage\vspace*{\fill}}
     {\vspace*{\fill}\par\pagebreak}

\sloppy

\pagestyle{fancy}

\setcounter{secnumdepth}{4}

\begin{document}

\begin{titlepage}

\vspace{-5.0cm}

%\begin{figure}[!htb]
 %\centering{\includegraphics[width=0.8\textwidth]{Figuras/rsz_logo_ifusp_fundo_claro.png}}
 %\label{fig:IFUSP_logo}
%\end{figure}

\begin{center}
\vspace{1cm}
%{\huge \textsf{Solicita\c{c}\~{a}o de Progress\~{a}o Funcional Docente}} \\[1cm]
\rule{1.0\textwidth}{1pt} \\ [0.5cm]
{\Huge \textbf{\textsf{Memorial Descritivo de Atividades}}} \\
\rule{1.0\textwidth}{1pt} \\
\vspace{2cm}

\doublespacing
{\Large \textsf{Solicita\c{c}\~{a}o de contratação para o cargo de de \textbf{Professor Doutor - MS-3.1} apresentada ao Instituto de Física "Gleb Wataghin", da Universidade de Campinas}}\\
\vspace{1.5cm}
{\LARGE \textsf{Solicitante: \textbf{André Monteiro Paschoal}}}\\
\vspace{0.5cm}
%{\Large \textsf{SIAPE: \textbf{1807586}}} \\
%\vspace{0.5cm}
%{\Large \textsf{Per\'{\i}odo: \textbf{20/08/2014 - 19/08/2016}}} \\

\vspace{2.0cm}

\normalsize \textsf{Abril de 2022}

\end{center}
\thispagestyle{empty}
\end{titlepage}


\tableofcontents
%\include{Lista_Anexos} \cleartooddpage[\thispagestyle{empty}]

%%%%%%%%%%%%%%%%%%%%%%%%%%%%%%%%%%%%%%%%%%%%%%%%%%%%%%%%%%%%%%%%%%%%%%%%%%%%%%%
% APRESENTA\c{C}\~{A}O
%%%%%%%%%%%%%%%%%%%%%%%%%%%%%%%%%%%%%%%%%%%%%%%%%%%%%%%%%%%%%%%%%%%%%%%%%%%%%%%

\newpage
\section*{Apresenta\c{c}\~{a}o}
\vspace{0.3cm}

\begin{onehalfspace}

Memorial apresentado por \textbf{André Monteiro Paschoal}, para concurso de Professor Temporário junto ao Instituto de Física da Universidade de São Paulo (IFUSP), para avalia\c{c}\~{a}o de desempenho acad\^{e}mico.

O presente memorial relata as atividades desempenhadas no per\'{\i}odo de \textbf{Fevereiro de 2013 até Setembro de 2020}. Os documentos comprobat\'{o}rios referenciados neste memorial est\~{a}o organizados em volumes anexos devidamente numerados. Por fim, os Anexos que se referem a artigos publicados em confer\^{e}ncias e peri\'{o}dicos cont\'{e}m apenas a primeira p\'{a}gina do trabalho e o email informando a aceita\c{c}\~{a}o do trabalhao para publica\c{c}\~{a}o (quando pertinente).

\end{onehalfspace}

%%%%%%%%%%%%%%%%%%%%%%%%%%%%%%%%%%%%%%%%%%%%%%%%%%%%%%%%%%%%%%%%%%%%%%%%%%%%%%%
% Grupo 1 - Identificação do Candidato
%%%%%%%%%%%%%%%%%%%%%%%%%%%%%%%%%%%%%%%%%%%%%%%%%%%%%%%%%%%%%%%%%%%%%%%%%%%%%%%
\newpage
\section{Identificação}

\begin{itemize}
        \large{\item \textbf{André Monteiro Paschoal.}}
        \item Filiação: \newline
        Pai: Oswaldo Luiz Fortes Paschoal. \newline
        Mãe: Ana Célia Navajas Monteiro Paschoal.
        \item Nascido em 04 de agosto de 1988 na cidade de Santa Cruz do Rio Pardo - SP.
        \item Cargo atual na carreira universitária: bolsista de Pós-Doutorado junto ao Instituto de Radiologia, Departamento de Radiologia e Oncologia do Hospital das Clínicas da Faculdade de Medicina da Universidade de São Paulo, sendo bolsista da Fundação de Amparo à Pesquisa do Estado de São Paulo (FAPESP).
        \item Residente na Avenida Santa Marina, (apartamento 171, torre 3) no bairro Água Branca no município de São Paulo - SP.
        \item Telefone para contato: (11) 98911-0265.
        \item E-mail para contato: ampaschoal@hmail.com 
        \item Membro das Seguintes Sociedades:
        \begin{itemize}
                \item International Society of Magnetic Resonance in Medicine (ISMRM).
                \item European Society of Magnetic Resonance in Medicine and Biology (ESMRMB).
                \item Membro colaborador da \textit{Open Science Iniciative for Perfusion Imaging} (OSIPI), líder da \textit{Task Force 6.1 - ASL Challenges}.
        \end{itemize} 
\end{itemize}

%%%%%%%%%%%%%%%%%%%%%%%%%%%%%%%%%%%%%%%%%%%%%%%%%%%%%%%%%%%%%%%%%%%%%%%%%%%%%%%
% Grupo 2 - Formação do Candidato
%%%%%%%%%%%%%%%%%%%%%%%%%%%%%%%%%%%%%%%%%%%%%%%%%%%%%%%%%%%%%%%%%%%%%%%%%%%%%%%
\newpage
\section{Formação}
\large{
\begin{enumerate}
        \item Graduação:
        \begin{itemize}
                \item Graduação em \textbf{Ciências Físicas e Biomoleculares}, obtida junto ao Instituto de Física de São Carlos (IFSC - USP) no ano de 2012. \mbox{\sffamily{\bfseries{[Doc. \ref{diplomas:graduacao}]}}} \\
        \end{itemize}

        \item Pós-graduação:
        \begin{itemize}
                \item Mestre em Ciências com grau obtido no programa de Física Aplicada: opção biomolecular, junto ao Instituto de Física de São Carlos (IFSC - USP) no ano de 2015. \mbox{\sffamily{\bfseries{[Doc. \ref{diplomas:mestrado}]}}} \\
                \item Doutor em Ciências com grau obtido no programa de Física Aplicada à Medicina e Biologia (FAMB), junto ao Departamento de Física da Faculdade de Filosofia, Ciências e Letras de Ribeirão Preto (FFCLRP) da Universidade de São Paulo (USP) no ano de 2020. \mbox{\sffamily{\bfseries{[Doc. \ref{diplomas:doutorado}]}}} \\
                \item Pós-doutorando junto à Faculdade de Medicina de Ribeirão Preto (FMRP-USP) desde Fevereiro de 2020. \mbox{\sffamily{\bfseries{[Doc. \ref{diplomas:posdoc1}]}}}
        \end{itemize}

        \item Idiomas:
        \begin{itemize}
                \item Inglês: compreende bem; lê bem; escreve bem; fala bem.
                \item Português: compreende bem; lê bem; escreve bem; fala bem.
        \end{itemize}
\end{enumerate}}

%%%%%%%%%%%%%%%%%%%%%%%%%%%%%%%%%%%%%%%%%%%%%%%%%%%%%%%%%%%%%%%%%%%%%%%%%%%%%%%
% Grupo 3 - Títulos
%%%%%%%%%%%%%%%%%%%%%%%%%%%%%%%%%%%%%%%%%%%%%%%%%%%%%%%%%%%%%%%%%%%%%%%%%%%%%%%
\newpage
\section{Títulos}
\subsection{Diplomas, dignidades universitárias e prêmios de cunho científico e cultural.}
\large{
\begin{enumerate}
        \item Prêmios de mérito científico:
        \begin{itemize}
                \item OHBM Travel Awards. \mbox{\sffamily{\bfseries{[Doc. \ref{awards:OHBM}]}}} \\
                \item Menção Honrosa no Prêmio CAPES de Teses 2021 \mbox{\sffamily{\bfseries{[Doc. \ref{awards:CAPES}]}}} \\
        \end{itemize}
\end{enumerate}}

%A seguir, listo as atividades de ensino que realizei no per\'{\i}odo, separadas por subgrupo, conforme rege o documento tal.

%%%%%%%%%%%%%%%%%%%%%%%%%%%%%%%%%%%%%%%%%%%%%%%%%%%%%%%%%%%%%%%%%%%%%%%%%%%%%%%
% Grupo 2: Atividades de Produ\c{c}\~{a}o Cient\'{\i}fica e T\'{e}cnica, Art\'{i}stica e Cultural
%%%%%%%%%%%%%%%%%%%%%%%%%%%%%%%%%%%%%%%%%%%%%%%%%%%%%%%%%%%%%%%%%%%%%%%%%%%%%%%
%\newpage

\subsection{\large{Participação  em  congressos,  simpósios  e  outros  certames  científicos  e  culturais  com apresentação  de  trabalhos}}
\vspace{0.3cm}

\begin{enumerate}
\renewcommand{\labelenumi}{{\large\bfseries\arabic{enumi}.}}

\item   \textbf{Evento:} III Semana Integrada do Instituto de Física de São Carlos. \mbox{\sffamily{\bfseries{[Doc. \ref{certificados:SIFSC2013}]}}} \\
        \textbf{Propósito:} Participante\\
        \textbf{Propósito (i):} Apresentação do resumo ``Otimização do Contraste em ASL Multifase''\\
        \textbf{Ano:} 2013\\
        \textbf{Local:} São Carlos - SP, Brasil.

\item   \textbf{Evento:} IV Semana Integrada do Instituto de Física de São Carlos \mbox{\sffamily{\bfseries{[Doc. \ref{certificados:SIFSC2014}]}}} \\
        \textbf{Propósito (i):} Participante\\
        \textbf{Propósito (ii):} Apresentação do resumo ``Otimização do Contraste em ASL Multifase''\\
        \textbf{Ano:} 2014\\
        \textbf{Local:} São Carlos - SP, Brasil.

        \item   \textbf{Evento:} I Transatlantic Workshop on Methods for Multimodal Neurosciences Studies. \mbox{\sffamily{\bfseries{[Doc. \ref{certificados:Transatlantic}]}}} \\
        \textbf{Propósito (i):} Participante\\
        \textbf{Propósito (ii):} Apresentação do resumo ``Contrast Optimization in Multiphase ASL.''\\
        \textbf{Ano:} 2014\\
        \textbf{Local:} São Pedro - SP, Brasil. 
        
        \item   \textbf{Evento:} V Semana Integrada do Instituto de Física de São Carlos. \mbox{\sffamily{\bfseries{[Doc. \ref{certificados:SIFSC2015}]}}} \\
        \textbf{Propósito (i):} Participante\\
        \textbf{Propósito (ii):} Apresentação do resumo ``Desenvolvimento de uma plataforma multimodal para o estudo da hemodinâmica cerebral: uma abordagem combinando ASL e NIRS.''\\
        \textbf{Ano:} 2015\\
        \textbf{Local:} São Carlos - SP, Brasil.

        \item   \textbf{Evento:} XV Semana da Física Médica. \mbox{\sffamily{\bfseries{[Doc. \ref{certificados:SFM2016}]}}} \\
        \textbf{Propósito (i):} Participante\\
        \textbf{Propósito (ii):} Apresentação do resumo ``Desenvolvimento de métodos para o estudo de funções e conectividade cerebral usando Arterial Spin Labeling.''\\
        \textbf{Ano:} 2016\\
        \textbf{Local:} Ribeirão Preto - SP, Brasil.

        \item   \textbf{Evento:} XXII Congresso Brasileiro de Física Médica. \mbox{\sffamily{\bfseries{[Doc. \ref{certificados:CBFM2017}]}}} \\
        \textbf{Propósito (i):} Participante\\
        \textbf{Propósito (ii):} Apresentação do resumo ``Arterial Transit Time Maps Utilizing Arterial Spin Labeling.''\\
        \textbf{Ano:} 2017\\
        \textbf{Local:} Ribeirão Preto - SP, Brasil.

        \item   \textbf{Evento:} Seminário semanal do programa de Física Aplicada à Medicina e Biologia. \mbox{\sffamily{\bfseries{[Doc. \ref{certificados:FAMB2017}]}}} \\
        \textbf{Propósito (i):} Palestrante\\
        \textbf{Propósito (ii):} Apresentação da palestra ``Improving Arterial Spin Labeling Acquisition to Reduce the Effect of Delayed Arrival Time.''\\
        \textbf{Ano:} 2017\\
        \textbf{Local:} Ribeirão Preto - SP, Brasil.

        \item   \textbf{Evento:} O Cérebro estatístico: desafios científicos do CEPID NeuroMat. \mbox{\sffamily{\bfseries{[Doc. \ref{certificados:Neuromat}]}}} \\
        \textbf{Propósito (i):} Ouvinte\\
        %\textbf{Propósito (ii):} Apresentação da palestra ``Improving Arterial Spin Labeling Acquisition to Reduce the Effect of Delayed Arrival Time.''\\
        \textbf{Ano:} 2017\\
        \textbf{Local:} Ribeirão Preto - SP, Brasil.

        \item   \textbf{Evento:} O 8º Simpósio de Instrumentação e Imagens Médicas (SIIM) e o 7º Simpósio de Processamento de Sinais (SPS). \mbox{\sffamily{\bfseries{[Doc. \ref{certificados:SIIM2017}]}}} \\
        \textbf{Propósito (i):} Participante \\
        \textbf{Propósito (ii):} Apresentação do resumo ``Evaluation of removing residual motion artifacts and global signal fluctuations in functional ASL data.''\\
        \textbf{Ano:} 2017\\
        \textbf{Local:} São Bernardo do Campo - SP, Brasil.

        \item   \textbf{Evento:} ISMRM 25th Annual Meeting \& Exhibition. \mbox{\sffamily{\bfseries{[Doc. \ref{certificados:ISMRM2017}]}}} \\
        \textbf{Propósito (i):} Participante \\
        \textbf{Propósito (ii):} Apresentação do resumo ``Improving Arterial Spin Labeling Acquisition to Reduce the Effect of Delayed Arrival Time.''\\
        \textbf{Ano:} 2017\\
        \textbf{Local:} Honolulu - Hawaii, EUA.

        \item   \textbf{Evento:} 4th BRAINN Congress \mbox{\sffamily{\bfseries{[Doc. \ref{certificados:BRAINN2017}]}}} \\
        \textbf{Propósito (i):} Ouvinte\\
        %\textbf{Propósito (ii):} Apresentação da palestra ``Improving Arterial Spin Labeling Acquisition to Reduce the Effect of Delayed Arrival Time.''\\
        \textbf{Ano:} 2017\\
        \textbf{Local:} Campinas - SP, Brasil.

        \item   \textbf{Evento:} XVI Semana da Física Médica. 
        \textbf{Propósito (i):} Participante\\
        \textbf{Propósito (ii):} Apresentação do resumo ``Brain Functional Analysis with Arterial Spin Labeling.'' \mbox{\sffamily{\bfseries{[Doc. \ref{certificados:SFM2017}]}}} \\
        \textbf{Propósito (iii):} Ministrante do Minicurso ``Processamento de Imagens Médicas'' \mbox{\sffamily{\bfseries{[Doc. \ref{certificados:SFM2017_avaliador}]}}} \\
        \textbf{Ano:} 2017\\
        \textbf{Local:} Ribeirão Preto - SP, Brasil.

        \item   \textbf{Evento:} 5th BRAINN Congress \mbox{\sffamily{\bfseries{[Doc. \ref{certificados:BRAINN2018}]}}} \\
        \textbf{Propósito (i):} Participante\\
        \textbf{Propósito (ii):} Apresentação do resumo ``Simultaneous assessment of CBF and brain function through Dual-Echo Arterial Spin Labeling.''\\
        \textbf{Ano:} 2018\\
        \textbf{Local:} Campinas - SP, Brasil.

        \item   \textbf{Evento:} Joint Annual Meeting ISMRM-ESMRMB. \mbox{\sffamily{\bfseries{[Doc. \ref{certificados:ISMRM2018}]}}} \\
        \textbf{Propósito (i):} Participante \\
        \textbf{Propósito (ii):} Apresentação do resumo ``Regularized nonnegative least-square fitting for intravoxel incoherent motion data processing: a simulation study.''\\
        \textbf{Propósito (iii):} Apresentação do resumo ``Brain connectivity assessment between rest condition and verbal fluency task through Arterial Spin Labeling.'' \\
        \textbf{Ano:} 2018\\
        \textbf{Local:} Paris, France.

        \item   \textbf{Evento:} XVII Semana da Física Médica. \mbox{\sffamily{\bfseries{[Doc. \ref{certificados:SFM2018}]}}} \\
        \textbf{Propósito (i):} Participante\\
        \textbf{Propósito (ii):} Apresentação do resumo ``A novel approach to delineate brain function and physiology under a semantic verbal fluency condition by a dual-echo ASL sequence.''\\
        \textbf{Ano:} 2018\\
        \textbf{Local:} Ribeirão Preto - SP, Brasil.

        \item   \textbf{Evento:} ISMRM Benelux Chapter. \mbox{\sffamily{\bfseries{[Doc. \ref{certificados:ISMRMBenelux}]}}} \\
        \textbf{Propósito (i):} Ouvinte\\
        %\textbf{Propósito (ii):} Apresentação da palestra ``Improving Arterial Spin Labeling Acquisition to Reduce the Effect of Delayed Arrival Time.''\\
        \textbf{Ano:} 2019\\
        \textbf{Local:} Leiden, the Netherlands.

        \item   \textbf{Evento:} Organization for Human Brain Mapping Annual Meeting. \mbox{\sffamily{\bfseries{[Doc. \ref{certificados:OHBM2019}]}}} \\
        \textbf{Propósito (i):} Participante \\
        \textbf{Propósito (ii):} Apresentação do resumo ``Organization Of Semantic Verbal Fluency Brain Network Assessed By Dual-Echo Arterial Spin Labeling.''\\
        \textbf{Ano:} 2019\\
        \textbf{Local:} Rome, Italy.

        \item   \textbf{Evento:} XVIII Semana da Física Médica. \mbox{\sffamily{\bfseries{[Doc. \ref{certificados:SFM2019}]}}} \\
        \textbf{Propósito (i):} Participante\\
        \textbf{Propósito (ii):} Apresentação do resumo ``Diffuse Glioma assessed by non-negative least square fitting for IVIM-MRI.''\\
        \textbf{Ano:} 2019\\
        \textbf{Local:} Ribeirão Preto - SP, Brasil.

        \item   \textbf{Evento:} MRTrix3 Workshop. \mbox{\sffamily{\bfseries{[Doc. \ref{certificados:mrtrix3}]}}} \\
        \textbf{Propósito (i):} Ouvinte\\
        \textbf{Ano:} 2020\\
        \textbf{Local:} Ribeirão Preto - SP, Brasil.

        \item   \textbf{Evento:} InBrain Workshop 2020: Advanced Brain Imaging. \mbox{\sffamily{\bfseries{[Doc. \ref{certificados:inbrain2020}]}}} \\
        \textbf{Propósito (i):} Palestrante\\
        \textbf{Ano:} 2020\\
        \textbf{Local:} Ribeirão Preto - SP, Brasil.

        \item   \textbf{Evento:} ISMRM 28th Annual Meeting \& Exhibition. \mbox{\sffamily{\bfseries{[Doc. \ref{certificados:ISMRM2020}]}}} \\
        \textbf{Propósito (i):} Participante \\
        %\textbf{Propósito (ii):} Apresentação do resumo ``Improving Arterial Spin Labeling Acquisition to Reduce the Effect of Delayed Arrival Time.''\\
        \textbf{Ano:} 2020\\
        \textbf{Local:} Virtual.

        \item   \textbf{Evento:} ESMRMB 37th Annual Meeting \& Exhibition. \mbox{\sffamily{\bfseries{[Doc. \ref{certificados:ESMRMB2020}]}}} \\
        \textbf{Propósito (i):} Participante \\
        \textbf{Propósito (ii):} Apresentação do resumo ``Evaluation of IVIM-MRI acquisition parameters for clinical protocol optimization in high-grade glioma patients.''\\
        \textbf{Ano:} 2020\\
        \textbf{Local:} Virtual.

        \item   \textbf{Evento:} ISMRM 29th Annual Meeting \& Exhibition. \mbox{\sffamily{\bfseries{[Doc. \ref{certificados:ISMRM2021}]}}} \\
        \textbf{Propósito (i):} Participante \\
        \textbf{Propósito (ii):} Apresentação do resumo ``The utility of IVIM maps in the assessment of microvascular perfusion of brain glioma.''\\
        \textbf{Propósito (iii):} Apresentação do resumo ``The Open Source Initiative for Perfusion Imaging (OSIPI) ASL MRI Challenge. In: Annual Meeting of the International Society of Magnetic Resonance in Medicine.''\\
        \textbf{Propósito (iv):} Apresentação do resumo ``Gaussian Mixture for Peak Identification in Non-Negative Least Squares Fitting of the IVIM Signal.''\\
        \textbf{Ano:} 2021\\
        \textbf{Local:} Virtual.

        \item   \textbf{Evento:} Jornada Paulista de Radiologia 2021. \mbox{\sffamily{\bfseries{[Doc. \ref{certificados:JPR2021}]}}} \\
        \textbf{Propósito (i):} Palestrante \\
        \textbf{Propósito (ii):} Apresentação da aula ``ASL: Física, Técnica, Sequências e Aplicações.''\\
        \textbf{Ano:} 2021\\
        \textbf{Local:} São Paulo - SP, Brasil.

        \item   \textbf{Evento:} ISMRM Perfusion Workshop: from head to toe. \mbox{\sffamily{\bfseries{[Doc. \ref{certificados:PWISMRM2022}]}}} \\
        \textbf{Propósito (i):} Palestrante e Participante \\
        \textbf{Propósito (ii):} Apresentação da palestra ``Results of the OSIPI Challenges (ASL).''\\
        \textbf{Propósito (iii):} Apresentação do resumo ``Feasibility of Arterial Spin Labeling to Assess Blood-Brain Barrier Permeability in Clinical Environment: Application to Multiple Sclerosis Patients.''\\
        \textbf{Ano:} 2022\\
        \textbf{Local:} Los Angeles - CA, USA.

\end{enumerate}

%\newpage

\subsection{\large{Obtenção  de  bolsa  de  estudo  em  instituições de  renome  científico  ou  cultural}}
\vspace{0.3cm}

\begin{enumerate}
\renewcommand{\labelenumi}{{\large\bfseries\arabic{enumi}.}}

\item   \textbf{Nível:} Mestrado. \mbox{\sffamily{\bfseries{[Doc. \ref{bolsas:mestradoCAPES}]}}} \\
        \textbf{Agência de Fomento:} CAPES\\
        \textbf{Período:} Março de 2013 à Junho de 2015\\
        \textbf{Local:} Instituto de Física de São Carlos - SP, Brasil.

\item   \textbf{Nível:} Doutorado \mbox{\sffamily{\bfseries{[Doc. \ref{bolsas:CNPQdoc}]}}} \\
        \textbf{Agência de Fomento} CNPq \\
        \textbf{Período:} Fevereiro de 2016 à Janeiro de 2020\\
        \textbf{Local:} Departamento de Física - Faculdade de Filosofia Ciências e Letras de Ribeirão Preto (FFCLRP - USP), Ribeirão Preto - SP, Brasil.

\item   \textbf{Nível:} Doutorado Sanduíche \mbox{\sffamily{\bfseries{[Doc. \ref{bolsas:PDSE}]}}} \\
        \textbf{Agência de Fomento:} PDSE - CAPES\\
        \textbf{Período:} Novembro de 2018 à Abril de 2019\\
        \textbf{Local:} Leiden University Medical Center, Leiden - the Netherlands. 

\item   \textbf{Nível:} Pós-Doutorado \mbox{\sffamily{\bfseries{[Doc. \ref{bolsas:PDJ}]}}} \\
        \textbf{Agência de Fomento:} PDJ - CNPq\\
        \textbf{Período:} Fevereiro de 2020 à Dezembro de 2020\\
        \textbf{Local:} Faculdade de Medicina de Ribeirão Preto (FMRP - USP), Ribeirão Preto - SP.

\end{enumerate}

%------------------------------------------------------------------------------
% TODO
\newpage
\section{Produção científica}
\subsection{Trabalhos aceitos em congressos}
\vspace{0.3cm}

\begin{enumerate}
\renewcommand{\labelenumi}{{\large\bfseries\arabic{enumi}.}}

\item 	PASCHOAL, A. M.; LEONI, R. ; SANTOS, A. ; FOERSTER, B. U. ; PAIVA, F. F.  \textbf{ASL Contrast Optimization in Multiphase STAR Labeling using Variable Flip Angle}. Organization for Human Brain Mapping, 2015, Honolulu. Proceedings of the Organization for Human Brain Mapping, 2015. v. 1. \textbf{[Doc. \ref{certificados:OHBM2015}]}

\item Paschoal, A.M.; dos Santos, A.C.; Paiva, F.F.; Leoni, R.F. \textbf{Non-negative Least Squares Fitting for IVIM-MRI in Diffuse Glioma}. Congresso Brasileiro de Física Médica, 2019. \textbf{[Doc. \ref{certificados:CBFM2019}]}

\item PASCHOAL, A.M.; SCHMID, S.; FRANKLIN, S.L.; LEONI, R.F.; OSCH, M.J.P.V. \textbf{3D GRASE readout optimization for time-encoded pCASL}. ESMRMB 2019, 36th Annual Scientific Meeting, 2019, Rotterdam, NL. October 3-5: Abstracts, Friday, 2019. v. 32. p. S152-S153. \textbf{[Doc. \ref{certificados:ESMRMB2019}]}

\end{enumerate}

%------------------------------------------------------------------------------
% TODO
\subsection{Artigos Científicos Publicados em Periódicos Nacionais e Internacionais na sua área de atuação}
\vspace{0.3cm}

\begin{enumerate}
        \renewcommand{\labelenumi}{{\large\bfseries\arabic{enumi}.}}
        
        \item PAIVA, F.F. ; FOERSTER, B.U. ; PASCHOAL, A.M. ; MOLL, F.F.T. ; MOLL NETO, J.N. \textbf{Otimização do Contraste em ASL Multi-fase}. Revista Brasileira de Física Médica (Online), v. 7, p. 41-44, 2013.  \textbf{[Doc. \ref{artigos:RBFM2013}]}
        
        \item Silva, J.P.S.; Monaco, L.M.; Paschoal, A.M.; Oliveira, I.A; Leoni, R.F. \textbf{Effects of global signal regression and subtraction methods on resting-state functional connectivity using arterial spin labeling data}. Magnetic Resonance Imaging, v 51, p 151-157, 2018. doi: 10.1016/j.mri.2018.05.006 \colorbox{yellow}{(Qualis A2)} \textbf{[Doc. \ref{artigos:MRI2018}]}
        
        \item Paschoal, A.M.; Leoni, R.F.; dos Santos, A.C.; Paiva, F.F. \textbf{Intravoxel incoherent motion MRI in neurological and cerebrovascular diseases}. NeuroImage-Clinical, v. 20, p. 705-714, 2018. doi: 10.1016/j.nicl.2018.08.030 \colorbox{yellow}{(Qualis A1)} \textbf{[Doc. \ref{artigos:NICLIN2018}]}
        
        \item Paschoal, A.M.; Paiva, F.F.; Leoni, R.F. \textbf{Dual-Echo Arterial Spin Labeling for Brain Perfusion Quantification and Functional Analysis}. CONCEPTS IN MAGNETIC RESONANCE PART A, v. 2019, p. 1-7, 2019. doi: 10.1155/2019/5040465 \textbf{[Doc. \ref{artigos:CONCEPTS2019}]}

        \item Paschoal, A.M.; Leoni, R.F.; Foerster, B.U.; dos Santos, A.C.; Pontes Neto, O.M.; Paiva, F.F. \textbf{Contrast optimization in arterial spin labeling with multiple post-labeling delays for cerebrovascular assessment}. Magnetic Resonance Materials in Physics, Biology and Medicine (MAGMA) v. online, 2020. doi: 10.1007/s10334-020-00883-z \colorbox{yellow}{(Qualis A2)} \textbf{[Doc. \ref{artigos:MAGMA2020}]}
        
        \item PASCHOAL, A.M.; LEONI, R.F.; PASTORELLO, B.F.; OSCH, M.J.P.; \textbf{Three-dimensional gradient and spin-echo readout for time-encoded pseudo-continuous arterial spin labeling: Influence of segmentation factor and flow compensation}. Magnetic Resonance in Medicine, v. 86, p. 1454-1462, 2021. doi: 10.1002/mrm.28807 \colorbox{yellow}{(Qualis A1)} \textbf{[Doc. \ref{artigos:MRM2021}]}
        
        \item PASCHOAL, A.M.; SILVA, P.H.R.; RONDINONI, C.; ARRIGO, I.V.; PAIVA, F.F.; LEONI, R.F.; \textbf{Semantic verbal fluency brain network: delineating a physiological basis for the functional hubs using dual-echo ASL and graph theory approach}. Journal of Neural Engineering, v. 18, p. 1-15, 2021. doi:10.1088/1741-2552/ac0864 \colorbox{yellow}{(Qualis A1)} \textbf{[Doc. \ref{artigos:JNE2021}]}
        
        \item PASCHOAL, A.M.; \textbf{Editorial for -Diffusion Tensor Imaging Reveals Altered Topological Efficiency of Structural Networks in Type-2 Diabetes Patients With and Without Mild Cognitive Impairment}. Journal of Magnetic Resonance Imaging, v.55, p. 928-929, 2021. doi:10.1002/jmri.27899 \colorbox{yellow}{(Qualis A1)} \textbf{[Doc. \ref{artigos:JMRI2021}]} 
        
        \item Paschoal, A.M.; Zotin, M.C.Z.; COSTA, L.M.; Santos, A.C.; Leoni, R.F.; \textbf{Feasibility of intravoxel incoherent motion in the assessment of tumor microvasculature and blood-brain barrier integrity: a case-based evaluation of gliomas}. Magnetic Resonance Materials in Physics, Biology and Medicine (MAGMA) v.35, p. 17-27, 2020. doi: 10.1007/s10334-021-00987-0 \colorbox{yellow}{(Qualis A2)} \textbf{[Doc. \ref{artigos:MAGMA2021}]}

\end{enumerate}

\subsection{Dissertação de Mestrado}
\vspace{0.3cm}

\begin{enumerate}
\renewcommand{\labelenumi}{{\large\bfseries\arabic{enumi}.}}

        \item PASCHOAL, André Monteiro. Otimização do contraste em Arterial Spin Labeling multifase. 2015. Dissertação (Mestrado em Física Aplicada) - Instituto de Física de São Carlos, University of São Paulo, São Carlos, 2015. doi:10.11606/D.76.2015.tde-29092015-101918. Acesso em: 2020-09-15. Disponível online em: \url{https://teses.usp.br/teses/disponiveis/76/76132/tde-29092015-101918/pt-br.php}

\end{enumerate}

\subsection{Tese de doutorado}
\vspace{0.3cm}

\begin{enumerate}
\renewcommand{\labelenumi}{{\large\bfseries\arabic{enumi}.}}

        \item PASCHOAL, André Monteiro. Optimization and application of quantitative magnetic resonance imaging methods to analyze brain perfusion and function. 2019. Tese (Doutorado em Física Aplicada à Medicina e Biologia) - Faculdade de Filosofia, Ciências e Letras de Ribeirão Preto, University of São Paulo, Ribeirão Preto, 2020. doi:10.11606/T.59.2020.tde-19032020-104417. Acesso em: 2020-09-15. Disponível online em: \url{https://www.teses.usp.br/teses/disponiveis/59/59135/tde-19032020-104417/pt-br.php}

\end{enumerate}

%------------------------------------------------------------------------------
% TODO
\newpage
\section{Experiências Didáticas Universitárias}
\subsection{Monitorias para disciplinas de graduação}
\vspace{0.3cm}

\begin{enumerate}
\renewcommand{\labelenumi}{{\large\bfseries\arabic{enumi}.}}

\item \textbf{Disciplina:} Laboratório de Física Geral II \textbf{[Doc. \ref{monitorias:IFSC}]}\\
      \textbf{Departamento:} Instituto de Física de São Carlos - IFSC - USP\\
      \textbf{Ano:} 2015\\
      \textbf{Programa:} Monitoria Institucional do IFSC.\\
      \textbf{Bolsa:} Bolsa do IFSC \\

\item \textbf{Disciplina:} Processamento de imagens médicas \textbf{[Doc. \ref{monitorias:FFCLRP}]}\\
      \textbf{Departamento:} Instituto de Física de São Carlos - IFSC - USP\\
      \textbf{Ano:} 2015\\
      \textbf{Programa:} Monitoria Institucional do IFSC.\\
      \textbf{Bolsa:} Bolsa do IFSC \\

\item \textbf{Disciplina:} Imagens por Ressonância Magnética Nuclear em Biomedicina \textbf{[Doc. \ref{monitorias:PAE1}]}\\
      \textbf{Departamento:} Departamento de Física - Faculdade de Filosofia, Ciências e Letras de Ribeirão Preto - FFCLRP - USP\\
      \textbf{Ano:} 2017\\
      \textbf{Programa:} Monitor do Programa de Aperfeiçoamento ao Ensino (PAE).\\
      \textbf{Bolsa:} Sem bolsa \\

\item \textbf{Disciplina:}  Física I (teórica vinculada) \\
       %\textbf{[Doc. \ref{project:2015-facepe-pepe}]}\\
      \textbf{Departamento:} Departamento de Física - Faculdade de Filosofia, Ciências e Letras de Ribeirão Preto - FFCLRP - USP \textbf{[Doc. \ref{monitorias:PAE2}]}\\
      \textbf{Ano:} 2019\\
      \textbf{Programa:} Monitor do Programa de Aperfeiçoamento ao Ensino (PAE).\\
      \textbf{Bolsa:} Bolsista PAE \\

\end{enumerate}

%------------------------------------------------------------------------------

\subsection{Membro de comissão avaliadora}
\vspace{0.3cm}

\begin{enumerate}
\renewcommand{\labelenumi}{{\large\bfseries\arabic{enumi}.}}

\item \textbf{Evento:} Simpósio Internacional de Iniciação Científica e Tecnológica da USP - SIICUSP. \textbf{[Doc. \ref{avaliador:SIICUSP2019}]}\\
      \textbf{Departamento:} Departamento de Física - Faculdade de Filosofia, Ciências e Letras de Ribeirão Preto - FFCLRP - USP\\
      \textbf{Ano:} 2019\\

\item \textbf{Evento:} XVIII Semana da Física Médica, 2019 \textbf{[Doc. \ref{avaliador:SFM2019}]}\\
      \textbf{Departamento:} Departamento de Física - Faculdade de Filosofia, Ciências e Letras de Ribeirão Preto - FFCLRP - USP\\
      \textbf{Ano:} 2019\\

\end{enumerate}
%------------------------------------------------------------------------------

%%%%%%%%%%%%%%%%%%%%%%%%%%%%%%%%%%%%%%%%%%%%%%%%%%%%%%%%%%%%%%%%%%%%%%%%%%%%%%%
% Subgrupo 3.2 - Coordenação de Eventos e Conferencista
%%%%%%%%%%%%%%%%%%%%%%%%%%%%%%%%%%%%%%%%%%%%%%%%%%%%%%%%%%%%%%%%%%%%%%%%%%%%%%%
\section{Coordenação de Eventos e Conferencista}
\vspace{0.3cm}

%------------------------------------------------------------------------------

\subsection{Comiss\~{a}o Organizadora de Eventos Internacional, Nacional, Regional ou Local}
\vspace{0.3cm}

\begin{enumerate}
\renewcommand{\labelenumi}{{\large\bfseries\arabic{enumi}.}}

    \item Program Committee Member of the InBrain Workshop 2020: Advanced Brain Imaging \textbf{[Doc. \ref{organizacao:inbrain}]}

    \item III Escola de Inverno em Física Aplicada à Medicina e Biologia (III EIFAMB). \textbf{[Doc. \ref{organizacao:eifamb}]}
    
\end{enumerate}

%%%%%%%%%%%%%%%%%%%%%%%%%%%%%%%%%%%%%%%%%%%%%%%%%%%%%%%%%%%%%%%%%%%%%%%%%%%%%%%
% LISTA DE ANEXOS
%%%%%%%%%%%%%%%%%%%%%%%%%%%%%%%%%%%%%%%%%%%%%%%%%%%%%%%%%%%%%%%%%%%%%%%%%%%%%%%

% Appendix
\clearpage
%\addappheadtotoc
\appendix
%\appendixpage
\newpage
\section{Documentos comprobatórios}
Esta seção contém os documentos comprobatórios referentes às atividades listadas neste memorial.
\addcontentsline{toc}{section}{Documentos comprobatórios}
\renewcommand{\thesubsection}{\arabic{subsection}}
% \renewcommand{\subsection}{
% \titleformat{\subsection}
%   {\Huge\bfseries\center\vspace{.4\textwidth}\thispagestyle{fancy}} % format
%   {}                % label
%   {0pt}             % sep
%   {\huge}           % before-code
% }

% !TEX root = memorial_amp.tex

%%%%%%%%%%%%%%%%%%%%%%%%%%%%%%%%%%%%%%%%%%%%%%%%%%%%%%%%%%%%%%%%%%%%%%%%%%%%%%%
% Grupo 1 - Atividades de Ensino
%%%%%%%%%%%%%%%%%%%%%%%%%%%%%%%%%%%%%%%%%%%%%%%%%%%%%%%%%%%%%%%%%%%%%%%%%%%%%%%

%%%%%%%%%%%%%%%%%%%%%%%%%%%%%%%%%%%%%%%%%%%%%%%%%%%%%%%%%%%%%%%%%%%%%%%%%%%%%%%
% Subgrupo 1.1 - Orienta\c{c}\~{o}es e Co-Orienta\c{c}\~{o}es
%%%%%%%%%%%%%%%%%%%%%%%%%%%%%%%%%%%%%%%%%%%%%%%%%%%%%%%%%%%%%%%%%%%%%%%%%%%%%%%

\newpage
\subsection{Diploma de Graduação em XXX, junto ao XXXX}
\label{diplomas:graduacao}
Esta subseção apresenta o diploma de graduação do candidato.
\includepdf[pages=-, scale=1,pagecommand=\thispagestyle{empty}]{\detokenize{Diplomas/Comprovante_Fake}}

\newpage
\subsection{Atestado de estágio de iniciação científica no Laboratório XXXX, localizado no 
Instituto XXXX}
\label{diplomas:IC}
Esta subseção apresenta o atestado de realização de iniciação científica no Laboratório XXXX, localizado no 
Instituto XXXXX, desenvolvendo o projeto "XXXX" sob orientação do Prof. 
Dr. XXXX no período de janeiro à dezembro de 2009.
\includepdf[pages=-, scale=1,pagecommand=\thispagestyle{empty}]{\detokenize{Diplomas/Comprovante_Fake}}

\newpage
\subsection{Prêmios de mérito científico}
\label{awards:OHBM}
Esta subseção apresenta o certificado de obtenção do Prêmio \textbf{OHBM Travel Award}, concedido no Encontro Anual da \textit{Organization for Human Brain Mapping} no ano de 2019, realizado em Roma - Itália.
\includepdf[pages=-, scale=1,pagecommand=\thispagestyle{empty}]{\detokenize{Diplomas/Comprovante_Fake}}

%%%%%%%%%%%%%%%%%%%%%%%%%%%%%%%%%%%%%%%%%%%%%%%%%%%%%%%%%%%%%%%%%%%%%%%%%%%%%%%
% Subgrupo 2.1 - Produtividade de Pesquisa
%%%%%%%%%%%%%%%%%%%%%%%%%%%%%%%%%%%%%%%%%%%%%%%%%%%%%%%%%%%%%%%%%%%%%%%%%%%%%%%

\newpage
\subsection{Participa\c{c}\~{a}o em Eventos Cient\'{\i}ficos (com apresenta\c{c}\~{a}o de trabalho ou oferecimento de cursos, palestras ou debates}
\label{certificados:SIFSC2013}
Esta subseção apresenta o comprovante da participação na XXXX do Instituto XXXX com seus respectivos propósitos.
\includepdf[pages=-, scale=1,pagecommand=\thispagestyle{empty}]{\detokenize{Diplomas/Comprovante_Fake}}

%---

\newpage
\subsection{Obtenção  de  bolsa  de  estudo  em  instituições de  renome  científico  ou  cultural}
\label{bolsas:mestradoCAPES}
Esta subseção apresenta o comprovante de concessão da bolsa de mestrado CAPES/PROEX.
\includepdf[pages=-, scale=1,pagecommand=\thispagestyle{empty}]{\detokenize{Diplomas/Comprovante_Fake}}

%---

\newpage
\subsection{Autoria de artigos completos publicados em anais de congresso, em jornais e revistas de circulação nacional e internacional na sua área}
\label{certificados:OHBM2015}
Esta subseção apresenta o comprovante da autoria do resumo publicado em anais do congresso Organization for Human Brain Mapping, 2015.
\includepdf[pages=-, scale=1,pagecommand=\thispagestyle{empty}]{\detokenize{Diplomas/Comprovante_Fake}}


%---

\newpage
\subsection{Autoria de artigos completos publicados em jornais e revistas de circulação nacional e internacional na sua área}
\label{artigos:MRI2018}
Esta subseção apresenta o comprovante da autoria do artigo publicado na revista nacional \textit{Revista Brasileira de Física Médica} no ano de 2013.
\includepdf[pages=-, scale=1,pagecommand=\thispagestyle{empty}]{\detokenize{docs/Comprovante_Fake}}
%---

\newpage
\subsection{Atividades de docência}
\label{aulas:EEP}
Esta subseção apresenta o certificado de atividades como docente no curso de Especialização em Ressonância Magnética para Biomédicos e Tecnólogos pela Escola de 
Educação Permanente do Hospital das Clínicas da Faculdade de Medicina da Universidade de São Paulo nos anos de 2021 e 2022, ministrando as disciplinas de física básica e física avançada.
\includepdf[pages=-, scale=1,pagecommand=\thispagestyle{empty}]{\detokenize{Diplomas/Comprovante_Fake}}

\newpage
\subsection{Monitorias para disciplinas de graduação}
\label{monitorias:IFSC}
Esta subseção apresenta o comprovante de Monitoria para alunos de graduação.
\includepdf[pages=-, scale=1,pagecommand=\thispagestyle{empty}]{\detokenize{Diplomas/Comprovante_Fake}}
%---

\newpage
\subsection{Membro de Comissão Avaliadora}
\label{avaliador:SIICUSP2019}
Esta subseção apresenta o comprovante de participação em comissão avaliadora de eventos.
\includepdf[pages=-, scale=1,pagecommand=\thispagestyle{empty}]{\detokenize{Diplomas/Comprovante_Fake}}
%---

%%%%%%%%%%%%%%%%%%%%%%%%%%%%%%%%%%%%%%%%%%%%%%%%%%%%%%%%%%%%%%%%%%%%%%%%%%%%%%%
% Subgrupo 3.2 - Coordenação de Eventos e Conferencista
%%%%%%%%%%%%%%%%%%%%%%%%%%%%%%%%%%%%%%%%%%%%%%%%%%%%%%%%%%%%%%%%%%%%%%%%%%%%%%%

%\newpage
%\subsection{Comiss\~{a}o Organizadora de Eventos Internacional, Nacional, Regional ou Local}
%\label{reviewer:2015-sbcars}
%Esta subseção apresenta o comprovante de Comiss\~{a}o Organizadora do 9th Brazilian Symposium on Software Components, Architectures and Reuse (SBCARS 2015).
%\includepdf[pages=-, scale=1,pagecommand=\thispagestyle{empty}]{\detokenize{GRUPO 3/Sub-Grupo 32/Comprovante Fake}}

%%%%%%%%%%%%%%%%%%%%%%%%%%%%%%%%%%%%%%%%%%%%%%%%%%%%%%%%%%%%%%%%%%%%%%%%%%%%%%%
% Grupo 4: Atividades de Forma\c{c}\~{a}o e Capacita\c{c}\~{a}o Acad\^{e}mica
%%%%%%%%%%%%%%%%%%%%%%%%%%%%%%%%%%%%%%%%%%%%%%%%%%%%%%%%%%%%%%%%%%%%%%%%%%%%%%%

\newpage
\subsection{Comiss\~{a}o Organizadora de Eventos Internacional, Nacional, Regional ou Local}
\label{organizacao:inbrain}
Esta subseção apresenta o comprovante de Participação de Comissão Organizadora de Eventos Nacionais e Internacionais.
\includepdf[pages=-, scale=1,pagecommand=\thispagestyle{empty}]{\detokenize{Diplomas/Comprovante_Fake}}


%------------------------------------------------------------------------------
%%%%%%%%%%%%%%%%%%%%%%%%%%%%%%%%%%%%%%%%%%%%%%%%%%%%%%%%%%%%%%%%%%%%%%%%%%%%%%%
% \newpage
% \subsection{Portaria de Progressão}
% \label{app:2014-portaria-progressao}
% \includepdf[pages=-, scale=1,pagecommand=\thispagestyle{empty}]{\detokenize{GRUPO 1/20141205_Portaria-de-Progressao-Funcional_5929-2014.pdf}}


\end{document}

%%% EOF
